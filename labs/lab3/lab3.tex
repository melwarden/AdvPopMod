\documentclass{article}
\title{MAR 508 Lab 3}
\author{Melissa Warden}
\date{\today}

\usepackage{graphicx}
\begin{document}
\maketitle
   
\section{Linear mixed effects using R}
\subsection{streams}

From Lecture\_03.r

\includegraphics[height = 5in]{fig1resids.pdf}

\subsection{weight-length}

\section{Materials and Methods}
\subsection{The data}
\subsubsection{NEFOP}
Can refer to Reference document to describe processing of Observer data. Include number of hauls observed for each gear type in the time period and season under study. Describe deletion of inshore hauls in GIS.
\subsubsection{Vessel Trip Report}
Describe filling in of missing values and deletion of inshore hauls in GIS.
\subsubsection{Commercial}
Used to bump up VTR landings. Not used for NC and VA.
\subsubsection{NCDMF}
North Carolina Trip Ticket landings replace the landings for North Carolina in the Commercial database. Describe deletion of inshore hauls and how much was kept from Pamlico Sound. Used to bump up observer data.
\subsubsection{VMRC}
VMRC data replaces landings for Virginia in the Commercial database for 2003-2006 because of unreliability of nemareas in the dealer data once it went to electronic reporting.

\subsection{Unit of effort}
Describe why landings are used. Outline relationship of landings to bycatch. Explain addition of .001 to all total landings to account for 0 catch.

\section{Results}

\end{document}
